\documentclass[journal,final,a4paper,twoside]{PS}

%%% Dieser Block ist dem Betreuer des Projektseminars vorbehalten
\usepackage{PS}             % Alle Definitionen �ber den Seitenstil (auf keinen Fall editieren!!)
\usepackage[T1]{fontenc}
\usepackage[latin1]{inputenc}

\def\lehrveranstaltung{PROJEKTSEMINAR ROBOTIK UND COMPUTATIONAL INTELLIGENCE}
\def\ausgabe{Vol.18,~SS~2018}
\setcounter{page}{1}        % Hier die Seitennummer der Startseite f�r Gesamtdokument festlegen

%%% Ab hier k�nnen Eintr�ge von den Teilnehmern des Projektseminars gemacht werden
%%% Wenn neben den LaTeX-Paketen aus der Datei PS.sty noch weitere gebraucht werden,
%%% so ist dies dringend mit dem Betreuer abzukl�ren!

\begin{document}
\newcommand{\euertitel}{Titel der Ausarbeitung}   % Titel hier eintragen!
\newcommand{\betreuer}{Akad.Titel Vorname Name }  % Betreuerdaten hier eintragen (mit einem Leerzeichen am Ende)!

\headsep 40pt
\title{\euertitel}
% Autorennamen in der Form "Vorname Nachname" angeben, alphabetisch nach Nachname sortieren,
% nach dem letzen Autor kein Komma setzen, sondern mit \thanks abschlie�en
\author{Autor~A,
        Autor~B,
        Autor~C
\thanks{Diese Arbeit wurde von \betreuer unterst�tzt.}}

\maketitle


\begin{Zusammenfassung}

\end{Zusammenfassung}
\vspace{6pt}

\begin{abstract}

\end{abstract}

\section{Einf�hrung}


\section{Grundlagen}
\label{sec:grundlangen}

\subsection{Modules of an auditory system}
Subsection text.

\subsection{HARK}
Subsubsection text.

\subsection{Sound source localization}
Subsubsection text.

\subsection{Sound source tracking}
Subsubsection text.

\subsection{Sound source separation}
Subsubsection text.

\subsection{Filter}
Subsubsection text.

\subsection{Integration of HARK in ROS}
Subsubsection text.


\section{Zusammenfassung}
\label{sec:zus}


\appendices
\section{Optionaler Titel}
Anhang eins.
\section{}
Anhang zwei.



\section*{Danksagung}
Wenn ihr jemanden danken wollt, der Euch bei der Arbeit besonders
unterst�tzt hat (Korrekturlesen, fachliche Hinweise,...), dann ist hier der daf�r vorgesehene Platz.

\begin{thebibliography}{1}
\bibitem{IEEEhowto:kopka}
H.~Kopka and P.~W. Daly, \emph{A Guide to {\LaTeX}}, 3rd~ed. Harlow, England: Addison-Wesley, 1999.
\bibitem{wissPraxis:DFG}
Deutsche Forschungsgemeinschaft, \emph{Vorschl�ge zur Sicherung guter wissenschaftlicher Praxis}, Denkschrift, Weinheim: Wiley-VCH, 1998.
\end{thebibliography}


\end{document}\grid
\grid
