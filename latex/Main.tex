\documentclass[journal,final,a4paper,twoside]{PS}

%%% Dieser Block ist dem Betreuer des Projektseminars vorbehalten
\usepackage{PS}             % Alle Definitionen �ber den Seitenstil (auf keinen Fall editieren!!)
\usepackage[T1]{fontenc}
\usepackage[latin1]{inputenc}

\def\lehrveranstaltung{PROJEKTSEMINAR ROBOTIK UND COMPUTATIONAL INTELLIGENCE}
\def\ausgabe{Vol.18,~SS~2018}
\setcounter{page}{1}        % Hier die Seitennummer der Startseite f�r Gesamtdokument festlegen

%%% Ab hier k�nnen Eintr�ge von den Teilnehmern des Projektseminars gemacht werden
%%% Wenn neben den LaTeX-Paketen aus der Datei PS.sty noch weitere gebraucht werden,
%%% so ist dies dringend mit dem Betreuer abzukl�ren!

\begin{document}
\newcommand{\euertitel}{Titel der Ausarbeitung}   % Titel hier eintragen!
\newcommand{\betreuer}{Akad.Titel Vorname Name }  % Betreuerdaten hier eintragen (mit einem Leerzeichen am Ende)!

\headsep 40pt
\title{\euertitel}
% Autorennamen in der Form "Vorname Nachname" angeben, alphabetisch nach Nachname sortieren,
% nach dem letzen Autor kein Komma setzen, sondern mit \thanks abschlie�en
\author{Autor~A,
        Autor~B,
        Autor~C
\thanks{Diese Arbeit wurde von \betreuer unterst�tzt.}}

\maketitle


\begin{Zusammenfassung}

\end{Zusammenfassung}
\vspace{6pt}

\begin{abstract}

\end{abstract}

\section{Einf�hrung}

In nature the ability to hear, or in other words the ability to gain informations about your environment by sound pro- cessing is an essential skill for many animals as well as for humans. Whether it is for hunting prey, for communication or for drawing attention to potential threads, audition can help solving a variety of different tasks. It?s reasonable to assume that an auditory system could improve the performance of autonomous car driving as well given that humans already using this ability in car traffic most of the time. Each car has a horn and it?s an important tool for police, fire and rescue services in many countries who are using them officialy to indicate an energency. Informations such as the road character, the condition of your car, ? squealing tires? could be obtained
by an auditory system to improve the decision making for autonomous driving. 
\\
Auditory systems for robots and sound source localization in particular has been subject of research for quite some time.  Valin et all. showed in 2003 that a mobile robot can localize different types of sound sources over a range of 3 meters with a precision of 3�  in real time using an array of 8 microphones ( https://ieeexplore.ieee.org/document/1248813/ ). By 2016 they were able to localize and track simultaneous different moving sound sources over a range of 7m using beamforming and particle filtering ( https://arxiv.org/abs/1602.08139 ). Liu et all. took a different approach with a biologically inspired spiking neural network for sound localisation ( $https://link.springer.com/chapter/10.1007/978-3-540-87559-8_41 $) in 2008. Their exeprimentals results showed that their model could localise a sound source from the azimuth (angle of incidence) -90 to 90 degree. In 2009 Murray et all. presented a hybrid architecture using cross-correlation and recurrent neural networks for accoustic tracking in robots ($ https://link.springer.com/chapter/10.1007/11521082_5 $).Using only two microphones, their model has shown comparable with the capabilities of the human auditory cortex with the azimuth localisation differing by an average of +-0.4�.  Murase et all. used an array of 8 microphones mounted on a mobile robot in order to track multiple moving speaker. Their two key ideas were to use beamforming to locate the sound sources  and to use a set of Kalman filters to track the non-linear movements of the speaker ($ https://www.isca-speech.org/archive/archive_papers/interspeech_2005/i05_0249.pdf $). The used filters had different history lengths in order to reduce errors under noisy and echoic environments. As a result, multiple moving speakers could be tracked successfully even when speakers and the mobile robot moved non-linearly. So far, most of those systems have in common that they are built to work in closed or may be crowded environments to interact with people. Focusing on auditive systems for cars, we find that Fazenda et all. demonstrated an acoustic based safety emergency vehicle detection for intelligent transport systems in 2009 ($ https://ieeexplore.ieee.org/document/5332788/ $). Based on a cross microphone array, they were capable of determining the incoming direction of a siren as a sound source. For their suggested array radius, their methods based on time delay estimation outperform those based on calculating the intensity at the microphone array. 


\section{Grundlagen}
\label{sec:grundlangen}

\subsection{Modules of an auditory system}
Subsection text.

\subsection{HARK}
Subsubsection text.

\subsection{Sound source localization}
Subsubsection text.

\subsection{Sound source tracking}
Subsubsection text.

\subsection{Sound source separation}
Subsubsection text.

\subsection{Filter}
Subsubsection text.

\subsection{Integration of HARK in ROS}
Subsubsection text.


\section{Zusammenfassung}
\label{sec:zus}


\appendices
\section{Optionaler Titel}
Anhang eins.
\section{}
Anhang zwei.



\section*{Danksagung}
Wenn ihr jemanden danken wollt, der Euch bei der Arbeit besonders
unterst�tzt hat (Korrekturlesen, fachliche Hinweise,...), dann ist hier der daf�r vorgesehene Platz.

\begin{thebibliography}{1}
\bibitem{IEEEhowto:kopka}
H.~Kopka and P.~W. Daly, \emph{A Guide to {\LaTeX}}, 3rd~ed. Harlow, England: Addison-Wesley, 1999.
\bibitem{wissPraxis:DFG}
Deutsche Forschungsgemeinschaft, \emph{Vorschl�ge zur Sicherung guter wissenschaftlicher Praxis}, Denkschrift, Weinheim: Wiley-VCH, 1998.
\end{thebibliography}


\end{document}\grid
\grid
